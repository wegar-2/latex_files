\documentclass[12pt]{article}
\usepackage{amssymb}
\usepackage{amsmath}
\usepackage{amsthm}

\DeclareMathOperator{\arccot}{arccot}

\begin{document}


\noindent Thesis (Leibniz formula for $\pi$):

$$ \sum_{n=0}^{\infty} \frac{(-1)^n}{2n+1} = \frac{\pi}{4} $$

\noindent Take:
$$ y = \arctan(x) $$
then:
$$ y' = \frac{1}{1+x^2} $$

\noindent For $-1 < x < 1$ we have:
\begin{eqnarray*}
\frac{1}{1+x^2} &= \displaystyle\frac{1}{1 - (-x^2)} = \displaystyle\sum_{n=0}^{\infty} (-1)^n x^{2n}
\end{eqnarray*}

\noindent Integrating both sides from $0$ to $u$:

\begin{equation*}
\int_{0}^{u} \frac{1}{1+x^2} = \arctan(u) = \sum_{n=0}^{\infty} \frac{(-1)^n}{2n+1} u^{2n+1}
\end{equation*}

\noindent It is clear that the formula above holds for $u \in (-1, 1)$. How about $u=1$?
Note that the sequence $(a_n)$ given by:

$$
a_n := \frac{(-1)^n}{2n+1} 
$$
is decreasing and bounded. Thus, by monotone convergence theorem the formula holds for $u=1$, too.
\\ \noindent Consequently, since $\arctan(1) = \pi/2$:
$$
\frac{\pi}{2} = \sum_{n=0}^{\infty} \frac{(-1)^n}{2n + 1}
$$
\end{document}