\documentclass[12pt]{article}
\usepackage{amssymb}
\usepackage{amsmath}
\usepackage{amsthm}

\DeclareMathOperator{\arccot}{arccot}

\begin{document}


\textbf{Thesis}:

$$ I = \int_{0}^{\infty} \frac{\ln x}{1+x^2} dx = \frac{5\pi^5}{64} $$

\textbf{Demonstration}. Use substitution $ u := \ln x$ to proceed:
\begin{equation*}
I = \int_{0}^{\infty} \frac{\ln x}{1+x^2} dx = \int_{-\infty}^{0} \frac{u^4 e^u}{1 + e^{2u}} du
\end{equation*}
Writing $1/(1+e^{2u})$ as geometric series since $e^u \leq 1$:
\begin{equation*}
I = \sum_{n=0}^{\infty} (-1)^n \int_{-\infty}^{0} u^4 e^{(2n+1)u} du
\end{equation*}
Substituting $v = -u$ yields:
\begin{equation*}
I = \sum_{n=0}^{\infty} (-1)^n \int_{0}^{\infty} v^{(5-1)} e^{-(2n+1)v} dv
\end{equation*}
Since for $\lambda > 0$:
\begin{equation*}
\int_{0}^{\infty} x^{p-1} e^{-\lambda x} dx = \frac{\Gamma(p)}{\lambda^p}
\end{equation*}

The integral equals:
\begin{equation*}
I = \sum_{n=0}^{\infty} \frac{(-1)^n}{(2n+1)^5} \Gamma(5) = 4! \sum_{n=0}^{\infty} \frac{(-1)^n}{(2n+1)^5}
\end{equation*}

The formula can be rewritten using Dirichlet's beta function as:
\begin{equation*}
I = 4! \beta(5)
\end{equation*}

It can be proved that the value of Dirichlet beta function at 5 is $5\pi^5 / 1536$, hence:
\begin{equation*}
I = 24 \cdot \frac{\pi^5}{1536} = \frac{5\pi^5}{64}
\end{equation*}

\end{document}