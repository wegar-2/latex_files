\documentclass[12pt]{article}

\usepackage[utf8]{inputenc}
\usepackage[T1]{fontenc}

\usepackage{amsmath}
\usepackage{amssymb}
\usepackage{amsthm}


\theoremstyle{plain}
\newtheorem{thm}{Theorem} % reset theorem numbering for each chapter

\theoremstyle{definition}
\newtheorem{defn}[thm]{Definition}
\newtheorem{lemma}[thm]{Lemma}
\newtheorem{exmp}[thm]{Example}
\begin{document}

\title{Introduction to Automata Theory [WiP!]}
\author{Artur Wegrzyn}
\maketitle

\abstract{This text is intended to contain notes for the online course 
\textit{Automata Theory} by Jeffrey Ullman\footnote{The course is offered on platform
edx.org (as of 2021-07-27)}. Since you can read these words please note that the content
 of this file is far from being in the final shape and that it has not yet been self-
 reviewed to the extent that makes the author comfortable. Hence, dear reader, if you 
 decide to do so, you are using this text at your own risk. }

\tableofcontents

\section{Graphs: Fundamentals}
Given that graphs can be used as an 'intuition of first resort' when thinking about 
automata, it might be useful to revise the relevant definitions in this area. \\
\indent Note, however, that I am \textbf{not} saying that 'automata are graphs'! 
Such a statement is simply not correct. All I want to do here is revising definitions 
related to graphs.
\\ \indent Throughout this section book \cite{intro_to_graph_theory} is used as a 
reference. However, the definitions are not quoted literally and might be rephrased 
at times to better draw attention to certain points that author considers interesting.

\begin{defn}
\textbf{(Graph)} Let $(V, E)$ be an ordered pair os sets, with $V$ being 
\textit{vertex set} and $E$ being \textit{edge set} and with $V \neq \emptyset$ (i.e.
it is required that $V$ is not an empty set). Moreover, elements of $E$ are required to 
be two-element subsets of $V$. Then, the pair $(V, E)$ is a graph.
\end{defn}

The following definitions offers some useful vocabulary for talking about graphs.

\begin{defn}
\textbf{(Join/connection, incidence, adjacency, isolation)} Let $(V, E)$ be a graph.
Let $e = \{x, y \}$ (by definition: $x, y \in V$) be an edge of this graph, 
i.e. $e \in E$. Then, the edge $e$ is said to \textit{join} or \textit{connect} the 
vertices $x$ and $y$. Furthermore, it is said of the vertices $x, y$ that they are 
\textit{incident} to the edge $e$. Any two edges that are incident to the same 
vertex are said to be \textit{adjacent}. Lastly, if there is a vertex that is 
not incident to any edge of a graph then it is called \textit{isolated}.
\end{defn}

It should come as not surprise that equality of graphs is given by the following 
definition:

\begin{defn}
\textbf{(Equality of graphs)} Consider two graphs $G_1 = (V_1, E_1)$ and
$G_2 = (V_2, E_2)$. Graphs $G_1$ and $G_2$ are said to be \textit{equal} if 
$V_1 = V_2$ and $E_1 = E_2$.
\end{defn}

\section{Building Blocks: Alphabet and Words}

This section is intended as a pool of the most fundamental concepts needed throughout the 
course for definitions of automata. 

\begin{defn} \textbf{(Alphabet)} Consider a non-empty, finite set of (distinct) symbols 
$\mathcal{A}$. Such a set is an alphabet.
\end{defn}

\begin{exmp}
Alphabet $\mathcal{A} = \{a, b\}$ is a set consisting of latin letters $a$ and $b$. Note 
that, crucially, the letters $a, b$ are used here as \textit{symbols} and they are 
\textbf{not} used to denote e.g. number (like in algebra, where you can have that, 
for instance, $a \in \mathbb{R} $).
\end{exmp}

\begin{defn}
\textbf{(Binary alphabet)} The set $\mathcal{B} = \{ 0, 1 \}$ consisting of symbols
$0$ and $1$ is called \textit{binary alphabet}.
\end{defn}

\begin{exmp}
It is quite natural now to fix an alphabet $\mathcal{A}$ and to start considering
 \textit{sequences} (finite or infinite) of symbols from the alphabet $\mathcal{A}$.
 To this end, let's define the set $A_n$ to be (for $n = 0, 1, 2, ...$):
\begin{equation*}
A_n = \{ 0, 1, ..., n-1, n \}
\end{equation*}
Taking binary alphabet $\mathcal{B} = \{0, 1 \}$ I take the set of all 
\textit{functions} from $A_n$ into $\mathcal{B}$ for $n=3$, i.e the set 
$B_3 := \mathcal{B}^{A_3}$. One of its subsets is:


\end{exmp}


\section{Deterministic Finite Automata}

\section{Bibliography}
Unsurprisingly, the main reference is \cite{intro_to_automata_theory}.

\bibliography{foundations_bibliography}
\bibliographystyle{plabbrv}
\end{document}
