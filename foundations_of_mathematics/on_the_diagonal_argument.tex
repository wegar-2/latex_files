\documentclass[12pt]{article}

\usepackage[english]{babel}
\usepackage[utf8]{inputenc}
\usepackage[T1]{fontenc}

\usepackage{amsmath}
\usepackage{amsthm}
\usepackage{amsfonts}

\theoremstyle{plain}
\newtheorem{thm}{Theorem} % reset theorem numbering for each chapter

\theoremstyle{definition}
\newtheorem{defn}[thm]{Definition}
\newtheorem{lemma}[thm]{Lemma}
\newtheorem{exmp}[thm]{Example}
\begin{document}
\title{Notes On the Diagonal Argument}
\author{Artur Wegrzyn}
\maketitle

\abstract{The purpose of this text is to store a number of facts concerning the diagonal
 argument. This note is primarily intended for author's own use. }

\tableofcontents

\section{Introduction to the Diagonal Argument}
This section discusses at more length the version of the diagonal argument that seems 
(at least to the author) the most insightful for those looking to understand the 
underlying \textit{idea}. This exposition is intended to be just an extended
rephrasing of the diagonal argument presented in the enlightening book by Smith
\cite{intro_godel_theorems}. \\

Consider the problem of equipollence of the set of natural numbers $\mathbb{N}$
and the set of infinite sequences of ones and zeros -
i.e. the set $\{0, 1 \}^{\mathbb{N}}$. 
Assume that the two sets are equipollent, i.e. that there exists a 
bijection\footnote{Note: the symbol $\rightarrow$ is taken to have different denotation
 than the symbol $\longrightarrow$. The former is a logical connective representing 
 logical conditional, whereas the latter is used in the definition of functions to 
 represent the transition from the domain into the codomain of a function. }
$f: \mathbb{N} \longrightarrow \{0, 1 \}^{\mathbb{N}}$. This means that the members 
of the set $\{ 0, 1 \}^{\mathbb{N}}$ can be enumerated. In other words, if we wrote
 down the values of $f$ for consecutive natural numbers:
\begin{equation}
\label{consecutive_f_values}
f(0), f(1), f(2), ..., f(n-1), f(n), f(n+1), ...
\end{equation}

we could be sure to find a certain 
$(x_n)_{n \in \mathbb{N}} \in \{ 0, 1 \}^{\mathbb{N}}$ 
somewhere in the list \eqref{consecutive_f_values}. \\
\indent As the next step, consider writing down the list of infinite binary sequences 
in a tabular format - as demonstrated in the table
\ref{tab:diagonal-arg-standard}. More specifically, suppose that each row of the table
 corresponds to a value of $f$ for some natural number. The value is an infinite binary 
 sequence, which is a mapping from the set of natural numbers into the set $\{ 0, 1 \}$.
  It is therefore tempting to number the columns of the table with consecutive natural
  numbers $0, 1, 2, ...$. The body of cells thus obtained can be considered
  to be an infinite $\mathbb{N} \times \mathbb{N}$ matrix $A$:
  $$A = [a_{m, n}]_{\mathbb{N} \times \mathbb{N}}$$
 Where $m$ points at $m$-th row of the matrix and $n$ points at $n$-th column of 
  the matrix (both indexes $m$ and $n$ start at zero).
\\ \indent It stands to reason that the $m$-th row of the matrix $A$ is the value 
of the function $f$ at $m$, i.e. an infinite binary sequence $f(m)$. 
Taking:
$$ f(m) = (a_{m, n})_{n \in \mathbb{N}} $$
it is clear that the element $a_{m ,n}$ is the $n$-th binary digit of the 
infinte binary sequence $f(m)$.

\begin{table}[]
\centering
\begin{tabular}{llcccccccccccc}
\textbf{n:} &
  \textbf{} &
  \textbf{0} &
  \textbf{1} &
  \textbf{2} &
  \textbf{3} &
  \textbf{4} &
  \textbf{5} &
  \textbf{6} &
  \textbf{7} &
  \textbf{8} &
  \textbf{9} &
  \textbf{10} &
  \textbf{...} \\ \cline{1-13}
\textbf{f(0):} &  & \textbf{1} & 1          & 0          & 0          & 0   & 1   & 0   & 1   & 1   & 1   & 0   & ... \\
\textbf{f(1):} &  & 1          & \textbf{0} & 1          & 0          & 1   & 1   & 1   & 1   & 0   & 1   & 1   & ... \\
\textbf{f(2):} &  & 1          & 1          & \textbf{0} & 0          & 1   & 1   & 0   & 0   & 1   & 0   & 0   & ... \\
\textbf{f(3):} &  & 1          & 1          & 0          & \textbf{0} & 0   & 0   & 1   & 1   & 1   & 0   & 0   & ... \\
...            &  & ...        & ...        & ...        & ...        & ... & ... & ... & ... & ... & ... & ... & ...
\end{tabular}
\caption{Putative bijection 
$f: \mathbb{N} \longrightarrow \{0, 1\}^{\mathbb{N}} $ written down.
Only first four values $f(0), f(1), f(2)$ and $f(3)$ 
are (partially) shown. }
\label{tab:diagonal-arg-standard}
\end{table}


\section{Generalized Form of the Diagonal Argument}

\section{Applications of Diagonal Argument to Select Equipollence Problems}

\section{Application of the Diagonal Argument to the Halting Problem}

\section{Bibliography}

\bibliography{foundations_bibliography}
\bibliographystyle{plabbrv}
\end{document}
