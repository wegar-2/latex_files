\documentclass[12pt]{article}

\usepackage[english,polish]{babel}
\usepackage[utf8]{inputenc}
\usepackage[T1]{fontenc}

\usepackage{amsmath}
\usepackage{amsthm}


\theoremstyle{plain}
\newtheorem{thm}{Theorem} % reset theorem numbering for each chapter

\theoremstyle{definition}
\newtheorem{defn}[thm]{Definition}
\newtheorem{lemma}[thm]{Lemma}
\newtheorem{exmp}[thm]{Example}
\begin{document}

\title{On the concept of function [WiP!!!]}
\author{Artur Wegrzyn}
\maketitle

\selectlanguage{english}

\abstract{The purpose of this article is to present the 
concept of function. The overarching aim to provide multiple approaches to thinking 
about how the properties of the function can be interpreted. Importantly, this note is 
meant primarily for author's own use so some steps in the reasonings might be skipped.}


\tableofcontents

\section{Introduction}

\section{Relations}

\section{Function - definition}

\section{Properties - set-theoretical approach}

\section{Properties - algebraic approach}

\section{Partial order vs strict order}

\section{Bibliography}
The concept of function is fundamental so it is covered in most of instroductory 
handbooks that cover foundations of mathematics, algebra and analysis. 
I would like to share subjective selection of bibliography in English and Polish here.
In my view, the book \cite{sets_logic_computation} is an especially good reference for 
those looking for both: general introduction to foundations of mathematics and 
friendly exposition of the concept of function.

\selectlanguage{polish}
\bibliography{foundations_bibliography}
\bibliographystyle{plabbrv}
\end{document}
