\documentclass[12pt]{article}

\usepackage[english,polish]{babel}
\usepackage[utf8]{inputenc}
\usepackage[T1]{fontenc}

\usepackage{amsmath}
\usepackage{amsthm}

\theoremstyle{plain}
\newtheorem{thm}{Theorem} % reset theorem numbering for each chapter

\theoremstyle{definition}
\newtheorem{defn}[thm]{Definition}
\newtheorem{lemma}[thm]{Lemma}
\newtheorem{exmp}[thm]{Example}
\begin{document}

\title{Notes On the Concepts of Relation and Function [WiP]}
\author{Artur Wegrzyn}
\maketitle

\selectlanguage{english}

\abstract{The purpose of this text is to store a number of facts related to the 
concepts of \textit{relation} and \textit{function}. This note is primarily intended for
 author's own use. }

\tableofcontents


\section{Existence of Cartesian product in ZFC}
Fundamentally, I work with ZFC (cf. \cite{suppes_axiomatic_set_theory}, Chapter X for 
more details; I am not going to introduce the full formalism here). 
\subsection{Existence of Ordered Pair}
Consider two non-empty sets $X$ and $Y$ and let $x, y$ be such that $x \in X, y \in Y$.  
\begin{itemize}
\item[1. ] By the Axiom of Pair it is possible to form the set $P$ given $x, y$:
\begin{equation*}
P = \{ x, y \}
\end{equation*}
\item[2. ] Take Kuratowski's definition of ordered pair $(x, y)$:
\begin{equation}
\label{ordered_pair_definition}
(x, y) = \{ \{ x \}, \{ x, y \} \}
\end{equation}
But how can we justify existance of the set $(x,y)$?
\item[3. ] Showing that \eqref{ordered_pair_definition} is allowed. Consider $x$ and 
apply the Axiom of Pair to two sets $a, b$ such that $a = b = x$. 
This yields the set $P_x$:
\begin{equation*}
P_x = \{ x, x \} = \{ x \}
\end{equation*}
\item[4. ] Now it has been established that the sets $\{ x \}$ and 
$ \{ x, y \}$ exist for $x \in X, y \in Y$. Hence, applying Axiom of Pair 
to $\{ x \}$ and $ \{ x, y \}$ one more time  yields a set $\Pi$:
\begin{equation*}
\Pi = \{ \{ x \}, \{ x, y \} \}
\end{equation*}
which is just \eqref{ordered_pair_definition}.
\end{itemize}
Thus, given non-empty sets $X$ and $Y$ and some of their members $x \in X$ and 
$y \in Y$, it is permitted to form the ordered pair $(x, y)$. 

\subsection{Existence of Cartesian product}
It remains to be demonstrated that the set of all ordered pairs
of members of $X$ and $Y$ also exists.
\begin{itemize}
\item[1. ] Take any $x \in X, y \in Y$ and form the ordered pair $(x, y)$.
\item[2. ] Apply Axiom of Pair to $a$ and $b$ such that $a = b = (x, y)$ to get 
the set $\{a, b \}$, i.e. the set $ \{ (x, y) \}$.
\item[3. ] Use Axiom of Union to Form the following set $U_y$:
\begin{equation*}
U_y = \bigcup_{x \in X} \{ (x, y) \}
\end{equation*}
\item[4. ] Use Axiom of union again, this time - to form the following set $C$:
\begin{equation}
\label{definition_cartesian_product}
C := \bigcup_{y \in Y} \bigcup_{x \in X} \{ (x, y) \} = \bigcup_{y \in Y} U_y
\end{equation}
\end{itemize}
It remains to be shown that the set $C$ indeed contains all the ordered pairs that
can be formed from the members of sets $X$ and $Y$. To this end, pick any
$x_0 \in X$ and $y_0 \in Y$ and form the orderd pair $(x_0, y_0)$.
\\ \indent The question still stands - is it the case that $(x_0, y_0) \in C$? 
By definition:
\begin{equation*}
(x_0, y_0) \in C 
\Leftrightarrow 
\exists y \in Y: (x_0, y_0) \in U_y
\end{equation*}
Consider $y = y_0$. Since:
\begin{equation*}
U_{y_0} = \bigcup_{x \in X} \{ (x, y_0) \}
\end{equation*}
it does indeed hold that $(x_0, y_0) \in C$.
\\ \indent It is now established that the set $C$ of all ordered pairs of the members
of the sets $X$ and $Y$ exists. This set is called \textit{Cartesian product} and is
 denoted $X \times Y$. To summarize:
 
\begin{defn}
\textbf{(Cartesian product of two sets)} Let $X, Y$ be nonempty sets. Then, the set
of all ordered pairs $(x, y)$ such that $x \in X, y \in Y$ is denoted $X \times Y$ and 
it is called the Cartesian product of sets $X$ and $Y$.
\end{defn} 

\section{Relations}
Let's keep the sets $X$ and $Y$ nonempty. It has been established that it is permitted
to form the Cartesian product $X \times Y$.

\begin{defn}
\textbf{(Relation)}
Given a Cartesian product $X \times Y$ of two nonempty sets $X$ and $Y$ any subset $R$ of
$X \times Y$ (i.e. $R \subset X \times Y$) is a relation.
\end{defn}
If $R \subset X \times Y$ then relation $R$ is called a relation \textit{over} 
sets $X$ and $Y$. 
If $R \subset X \times X$ then relation $R$ is called a relation \textit{in} set $X$. \\
\indent Suppose that $(x, y) \in R$. Then it is said that $x$ is in relation $R$ 
with $y$. Notation $xRy$ is also used to denote the fact that $(x, y) \in R$.

\begin{defn}
\textbf{(Properties of relations)} Let $R$ be a relation in set $X$. 
Relation R is:
\begin{itemize}
\item[1. ] reflexive $:\Leftrightarrow$ $\forall x \in X: xRx$ 
\item[2. ] anti-reflexive (or irreflexive) 
$:\Leftrightarrow$ $\forall x \in X: \lnot xRx$ 
\item[2. ] symmetric $:\Leftrightarrow \forall x, y \in X: xRy \rightarrow yRx$
\item[3. ] anti-symmetric
$:\Leftrightarrow \forall x, y \in X: xRy \rightarrow \lnot yRx$
\item[4. ] asymmetric 
\item[5. ] transitive
$:\Leftrightarrow$ $\forall x, y, z \in X: (xRy \wedge yRz) \rightarrow xRz$ 
\item[6. ] total (or connected)
\end{itemize}
\end{defn}


\section{Orders}
\subsection{Definitions of Various Kinds of Orders}
\subsection{Strict Order and Weak Order}
\section{Function - Definition}
\section{Inverse Function - Definition 1}
\section{Inverse Function - Definition 2}
\section{Composition of Functions}
\section{Important Application: Equipollence of Sets}

\section{Bibliography}
The concept of function is a fundamental one for mathematics in general,
 so it is discussed in most instroductory 
handbooks that cover foundations of mathematics, algebra and analysis. 
\\ \indent I would like to share subjective selection of bibliography in English 
and Polish here. In my view, the book \cite{sets_logic_computation} is an especially 
friendly reference for  those looking for both: general introduction to foundations of 
 mathematics and friendly exposition of the concept of function.

\bibliography{foundations_bibliography}
\bibliographystyle{plabbrv}
\end{document}
