\documentclass[12pt]{article}

%%\usepackage[english]{babel}
\usepackage{amsmath}
\usepackage{amssymb}
\usepackage{url}


\begin{document}

\title{A note on gaussian AR(1) process [WiP!!!]}
\author{Artur Wegrzyn}
\maketitle

\abstract{This note presents elementary properties of autoregressive AR(1) gaussian 
process. It is intended as a future reference for the author, hence not too much effort
 is made to ensure that propounding of all steps is sufficient to satisfy the tastes of 
 more fastidious readers. Also, notation is at times quite sloppy and is assumed to be 
 self-explanatory.}

\tableofcontents

\section{AR(1) process - overview}
Let $\{Y_t\}_{t=-\infty}^{\infty}$ be a weakly stationary stochastic process that obeys
the equation:
\begin{equation}
\label{ar1_def_eqtn}
Y_t = \phi Y_{t-1} + c + \epsilon_t
\end{equation}
with gaussian error term, i.e.:
\begin{equation}
\epsilon_t \sim N(0, \sigma^2)
\end{equation}
and all the error terms being uncorrelated. Furthermore, suppose that $T+1$ consecutive 
realizations of the process $\{Y_t\}$ are observed for times $t = 0, 1, ..., T$ and that
 they are denoted $y_0, y_1, y_2, ..., y_T$. \\
 

\section{Likelihood function}
I start by looking at the conditional distribution of $Y_t$ with respect to $Y_{t-1}$. 
To this end, suppose that the value of $Y_{t-1}$ is known and equal to $y_{t-1}$. Then:
\begin{equation*}
Y_t = c + \phi y_{t-1} + \epsilon_t
\end{equation*}
Unsurprisingly, then:
$$ E(Y_t | Y_{t-1} = y_{t-1}) = c + \phi y_{t-1} $$
$$Var(Y_t | Y_{t-1} = y_{t-1}) = \sigma^2 $$
As $\{\epsilon_t\}$ has been assumed to be gaussian, it follows that the conditional 
probability density function of $Y_t$ with respect to $Y_{t-1} = y_{t-1}$ is given by:
\begin{equation}
\label{ar_cond_density}
f_{Y_t | Y_{t-1}}(y_t | y_{t-1}) = \frac{1}{\sigma \sqrt{2 \pi}}
\exp \left( - \frac{1}{2} \frac{(y_t - c - \phi y_{t-1})^2}{\sigma^2} \right)
\end{equation}
This formula will be of great use in the next step which is rewriting of joint 
probability density function of $(Y_T, Y_{T-1}, ..., Y_1, Y_0)$. 

Consequently, let's deal with the joint density. Using the simple fact presented in
 \cite{my_note_cond_prob_tsa}, I have: 

\begin{equation*}
\begin{split}
f_{Y_T, Y_{T-1}, ..., ,Y_1, Y_0}(y_T, y_{T-1}, ..., y_1, y_0) = 
& f_{Y_T | Y_{T-1}, ..., Y_1, Y_0}(y_T | y_{T-1}, ..., y_1, y_0) \cdot \\
& f_{Y_{T-1} | Y_{T-2}, ..., Y_1, Y_0}(y_{T-1} | y_{n-2}, ..., y_1, y_0) \cdot ... \cdot \\
& f_{Y_1 | Y_0} (y_1 | y_0) \cdot f_{Y_0}(y_0)
\end{split}
\end{equation*}

Written more succintly:
\begin{equation*}
f_{Y_T, Y_{T-1}, ..., ,Y_1, Y_0}(y_T, y_{T-1}, ..., y_1, y_0) = 
f_{Y_0}(y_0) \cdot
 \prod_{t=1}^{T} f_{Y_t | Y_{t-1}, ..., y_0} (y_t | y_{t-1}, ..., y_0)
\end{equation*}

Since I aim for \textit{conditional} likelihood, I devide both sided by $f_{Y_0}(y_0)$ 
and obtain:

\begin{equation*}
\label{full_cond_density}
f_{Y_T, Y_{T-1}, ..., ,Y_1 | Y_0}(y_T, y_{T-1}, ..., y_1 | y_0) = 
 \prod_{t=1}^{T} f_{Y_t | Y_{t-1}, ..., y_0} (y_t | y_{t-1}, ..., y_0)
\end{equation*}

Now, by equation \eqref{ar_cond_density} one can easily see that since the value of the
 process at time $t$ is conditional only on its value at time $t-1$ the conditional 
 densities in the product in the equation above reduce as follows:

\begin{equation} 
\label{reduction_cond_density}
f_{Y_t | Y_{t-1}, Y_{t-2} ..., Y_0}(y_t | y_{t-1}, y_{t-2}, ..., y_0) = 
f_{Y_t | Y_{t-1}} (y_t | y_{t-1})
\end{equation}

The intuitive explanation of the above is that by \eqref{ar_cond_density} the density of 
$Y_t$ is conditional on $Y_{t-1}$ only and not on $Y_{t-2}, Y_{t-3}, ...$. More formal
 approach to this topic is presented in one of the later sections. \\
For now, let's focus on obtaining an alaytical formula for the conditional likelihood 
function. Substituting \eqref{reduction_cond_density} into \eqref{full_cond_density} 
yields:

\begin{equation}
\label{cond_density_full_analytical_1}
f_{Y_T, Y_{T-1}, ..., ,Y_1 | Y_0}(y_T, y_{T-1}, ..., y_1 | y_0) = 
\prod_{t=1}^{T} \frac{1}{\sigma \sqrt{2\pi}} \exp \left(
-\frac{1}{2} \frac{(y_t - c - y_{t-1})^2}{\sigma^2}
\right)
\end{equation}

Since we consider the above as a function of the model parameters 
$\phi, c, \sigma$ conditional on the sample $y_0, y_1, ..., y_T$ I introduce 
the conditional lieklihood $L(\phi, c, \sigma | y_T, t_{T-1}, ..., y_1, y_0)$ as defined 
using the equation \eqref{cond_density_full_analytical_1} :

\begin{equation}
\label{cond_density_full_analytical_2}
 = 
\prod_{t=1}^{T} \frac{1}{\sigma \sqrt{2\pi}} \exp \left(
-\frac{1}{2} \frac{(y_t - c - y_{t-1})^2}{\sigma^2}
\right)
\end{equation}

\section{Derivation of condtional maximum likelihood estimators of the parameters}
This section contains transformations of the formula...

\section{Rewriting AR(1) as MA($\infty$)}
Let's go back to square one and aim for another interesting feature of AR(1) process.
Start with the equation \eqref{ar1_def_eqtn}:
\begin{equation*}
Y_t = c + \phi y_{t-1} + \epsilon_t
\end{equation*}

Writing this equation for time $t-1$:
\begin{equation*}
Y_{t-1} = c + \phi y_{t-2} + \epsilon_{t-1}
\end{equation*}
Substitution of the formula for $Y_{t-1}$ into the formula for $Y_t$ yields:
\begin{eqnarray*}
Y_t = & c + \phi \left(
c + \phi y_{t-2} + \epsilon_{t-1}
\right) + \epsilon_t \\
= & c + \phi c + \epsilon_t + \phi \epsilon_{t-1} + \phi^2 y_{t-2}
\end{eqnarray*}
Iterating one more time I obtain:
\begin{equation*}
Y_t =  c(1 + \phi + \phi^2) +
(\epsilon_t + \phi \epsilon_{t-1} + \phi^2 \epsilon_{t-2}) +
\phi^3 Y_{t-3}
\end{equation*}
Generalizing:
\begin{equation*}
Y_t =  c \sum_{k=0}^{n}\phi^k +
\sum_{k=0}^{n} \phi^k \epsilon_{t-k} +
\phi^{n+1} Y_{t-n-1}
\end{equation*}
Since $|\phi| < 1$, it follows that $\lim_{n \rightarrow \infty} \phi^{n+1} = 0$.
Therefore\footnote{I deliberately restrain from intruducing the polynomial notation here
to keep focus of what is of greatest interest here. }:

\begin{equation*}
\label{ar1_ma_representation}
Y_t =  \frac{c}{1 - \phi} +
\sum_{k=0}^{\infty} \phi^k \epsilon_{t-k}
\end{equation*}


\section{Unconditional distribution of $Y_t$}
I continue the analysis in the vein of \eqref{ar1_ma_representation}. Since 
all $\epsilon_t$ are normally, independently distributed with 
$\epsilon_t \sim N(0, \sigma^2)$, then the $Y_t$ is also normally distributed
with mean:
\begin{equation*}
E(Y_t) = \frac{c}{1 - \phi}
\end{equation*}
and variance:
\begin{eqnarray*}
Var(Y_t) = & Var(\sum_{n=0}^{\infty} \phi^n \epsilon_{t-n}) \\
= & \sum_{n=0}^{\infty} \phi^{2n} \sigma^2 \\
= & \frac{\sigma^2}{1 - \phi^2}
\end{eqnarray*}

This allows me to write down the unconditional density of $Y_t$:
\begin{equation}
\label{unconditional_density}
f_{Y_t}(y_t) = \frac{1}{\sqrt{2 \pi} (\sigma / \sqrt{1-\phi^2})}
\exp \left( 
- \frac{1}{2} 
\frac{(y_t - c/(1-\phi))^2}{\sigma^2/(1-\phi^2)}
\right)
\end{equation}


\section{Joint density of $Y_{t+2}, Y_{t+1}, Y_t$}
Building on top of the results of two previous sections I aim at deriving the joint 
density of $\{Y_t\}$ at three consecutive time instances, i.e. $Y_{t+2}, Y_{t+1}, Y_t$.
First, joint density of $Y_{t+1}, Y_t$ is by substitution of
 \eqref{unconditional_density} and \eqref{ar_cond_density} into the formula:
\begin{equation*}
f_{Y_{t+1}, Y_t} (y_{t+1}, y_t) = f_{Y_{t+1} | Y_t} (y_{t+1} | y_t) f_{Y_t}(y_t)
\end{equation*}
which yields:
\begin{eqnarray*}
\label{bivariate_density}
f_{Y_{t+1}, Y_t}(y_{t+1}, y_t) =
\frac{\sqrt{1 - \phi^2}}{\sigma^2 \sqrt{2 \pi}}
\exp \left(
-\frac{(y_t - c/(1-\phi))^2
(y_{t+1} - c - \phi y_t)^2}
{2\sigma^2(1- \phi^2)} 
\right)
\end{eqnarray*}
attack on the joint density of $Y_{t+2}, Y_{t+1}, Y_t$ proceeds analogously via:
\begin{equation*}
f_{Y_{t+2}, Y_{t+1}, Y_t} (y_{t+2}, y_{t+1}, y_t) = 
f_{Y_{t+2} | Y_{t+1}, Y_t} (y_{t+2} | y_{t+1}, y_t) f_{Y_{t+1}, Y_t}(y_{t+1}, y_t)
\end{equation*}
But how is the condtional density $f_{Y_{t+2} | Y_{t+1}, Y_t}$ obtained? 
To get it, let's reconsider the following relationships that express $Y_{t+2}$ as 
a function of $Y_{t+1}$ or a function of $Y_t$:
\begin{equation*}
Y_{t+2} = c + \phi Y_{t+1} + \epsilon_{t+2}
\end{equation*}
\begin{equation*}
Y_{t+2} = c + \phi c + \epsilon_{t+2} + \phi \epsilon_{t+1} + \phi^2 Y_t
\end{equation*}

First of these equations allows me to establish the following fact:
\begin{equation*}
Y_{t+2} | Y_{t+1} = y_{t+1} , Y_t = y_t \sim N(c + \phi y_{t+1}, \sigma^2)
\end{equation*}

Note that dependency of $Y_{t+2}$ on $Y_t$ is, by definition, already contained inside 
$Y_{t+1}$. Hence, the conditional density of interest is:

\begin{equation*}
f_{Y_{t+2} | Y_{t+1}, Y_t} (y_{t+2} | y_{t+1}, y_t) = 
\frac{1}{\sigma \sqrt{2 \pi}} 
\exp \left(
-\frac{1}{2} \frac{(y_{t+2} - c - \phi y_{t+1})^2}{\sigma^2}
\right)
\end{equation*}

And the formula for the joint density of $f_{Y_{t+2}, Y_{t+1}, Y_t} (y_{t+2}, y_{t+1},
 y_t)$ follows:

\begin{eqnarray*}
\label{threevariate_density}
f_{Y_{t+2}, Y_{t+1}, Y_t} (y_{t+2}, y_{t+1}, y_t)& =
\frac{\sqrt{1 - \phi^2}}{\sigma^3 (2 \pi)^{3/2}} \cdot \\
& \exp \left(
-\frac{(y_t - c/(1-\phi))^2
(y_{t+1} - c - \phi y_t)^2
(y_{t+2} - c - \phi y_{t+1})^2}
{2\sigma^3(1- \phi^2)} 
\right)
\end{eqnarray*}

\section{Unconditional maximum likelihood}


\section{References}
The exposition in this note follows closely the one in \cite{hamilton_tsa}, chapter 5. 
Another useful reference is \cite{fuller_intro_tsa}.

\bibliography{tsa_bibliography}
\bibliographystyle{ieeetr}

\end{document}
